\documentclass[11pt,a4paper]{article}
\usepackage[T1]{fontenc}
\usepackage{isabelle,isabellesym}
\usepackage{amsmath}
\usepackage{amssymb}
\usepackage{stmaryrd}
\usepackage{isabelle,isabellesym}

% this should be the last package used
\usepackage{pdfsetup}

% urls in roman style, theory text in math-similar italics
\urlstyle{rm}
\isabellestyle{it}


\begin{document}

\title{Generalised Arithmetical Operations in Higher-Order Tarski-Grothendieck Set Theory}
\author{Linghan Fang}
\maketitle

\begin{abstract}
In this work, we present the formalization of various arithmetical operations in
higher-order Tarski-Grothendieck set theory, as implemented in the Isabelle/Set
framework\footnote{\url{https://github.com/fangcq/Isabelle-Set}}.
We define generalized set addition and multiplication within said framework,
following the pen-and-paper work of Kirby~\cite{kirby_set_arithemtics}
and a prior formalization by Paulson~\cite{ZFC_in_HOL_AFP}.
We prove various properties, such as associativity laws, monotonicity rules, and injectivity laws.

Unlike Paulson, our formalization is based on a naive axiomatization of set theory within the
higher-order logic of Isabelle, whereas Paulson's work is based on the postulation of a constant
$\textit{elts} :: \textit{zfc\_set} \Rightarrow \textit{zfc\_set set}$
that embeds the type of ZFC sets within the type
of Isabelle/HOL sets.

We also define the concepts of set-orders, ordinals, cardinals as well as addition and
multiplication on cardinals. We demonstrate the compatibility between generalized set and cardinal
addition. The compatibility between generalized set and cardinal multiplication is left as future
work.
\end{abstract}

\tableofcontents

% include generated text of all theories
\input{session}

\bibliographystyle{abbrv}
\bibliography{root}

\end{document}
